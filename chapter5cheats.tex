\documentclass[10pt,twocolumn]{article}
\usepackage{listings,amssymb,amsmath,amsthm}
\usepackage{lingmacros}
\usepackage{tree-dvips}

\pdfpagewidth 8.5in
\pdfpageheight 11in 

\newenvironment{packed_list}{
\begin{itemize}
  \setlength{\itemsep}{1pt}
  \setlength{\parskip}{0pt}
  \setlength{\parsep}{0pt}
}{\end{itemize}}

\pagestyle{empty}

\setlength\topmargin{-.625in}
\setlength\headheight{0in}
\setlength\headsep{0in}
\setlength\textheight{10in}
\setlength\textwidth{7.75in}
\setlength\oddsidemargin{-.625in}
\setlength\evensidemargin{.25in}
\setlength\columnsep{.25in}
\setlength\footskip{0.5in}

\begin{document}
	\paragraph*{Analyzing Algorithms}
	The Optimal Algorithm Problem: Suppose algorithm $A$ solves
        problem $P$. Is $A$ the best solution to $P$?
	
	\paragraph*{Definition of Optimal in the Worst Case} An algorith
        $A$ is \emph{optimal in the worst case} for problem $P$, if
        for any algorithm $B$ that exists, or ever will exist, the
        following relationship holds:
	\begin{center}$W_A(n)\leq W_B(n),\forall n>0.$\end{center}
	
	\paragraph*{Decision Trees}
	Lower bound on a k-ary decision tree of depth n: $\lceil log_k n\rceil$
	
	\paragraph*{Summations Facts}
        \begin{packed_list}
	\item[a.] $ \sum_{k=m}^n \mathrm{c} = (n - m + 1)\mathrm{c}$ 
	\item[b.] $ \sum_{k=m}^n \mathrm{c a_\mathit{k}} = c
	\sum_{k=m}^n \mathrm{a_\mathit{k}}$
	\item[c.] $\sum_{k=1}^n (a_\mathit{k} - a_\mathit{k - 1}) =
	a_\mathit{n} - a_\mathit{0}$ 
		and $ \sum_{k=1}^n
	(a_\mathit{k - 1} - a_\mathit{k}) = a_\mathit{0} - a_\mathit{n}$
	\item[d.] $\sum_{k=m}^n (a_\mathit{k} + b_\mathit{k}) =
	\sum_{k=m}^n a_\mathit{k} + \sum_{k=m}^n
	b_\mathit{k}$
	\item[e.] $\sum_{k=m}^n a_\mathit{k} = \sum_{k=m}^i
	a_\mathit{k} +  \sum_{k=i+1}^n a_\mathit{k}$
	\item[f.] $\sum_{k=m}^n a_\mathit{k}x^\mathit{k+i} = 
	x^\mathit{i} \sum_{k=m}^n a_\mathit{k}x^\mathit{k}$
	\item[g.] $\sum_{k=m}^n a_\mathit{k+i} = \sum_{k=m+i}^{n+1}
	a_\mathit{k}$
        \end{packed_list}

	\paragraph*{Sum of Powers}
	When doing powers of sums we can use the form of $k^m$ can also be
	represented as $k^{m+1} - ( k + 1)^{m+1}.$

	\subsubsection*{Closed Forms of Elementary Finite Sums}
	\begin{packed_list}
        \item[a.] $\sum_{k=1}^n k = \frac{n(n+1)}{2}$
	\item[b.] $\sum_{k=1}^n k^2 = \frac{n(n+1)(2n+1)}{6}$
	\item[c.] $\sum_{k=0}^n a^k = \frac{a^{n+1} - 1}{a -1} ( a \neq
	1)$
	\item[d.] $
\sum_{k=1}^n ka^k = \frac{a - (n + 1)a^{n+1} +
	  na^{n+2}}{(a-1)^2} (a \neq 1)$
        \end{packed_list}

	\subsubsection*{Abel's Summation Tranformation}
	$\sum_{k=0}^n a_kb_k = \mathit{A}_nb_n + \sum_{k=0}^{n-1}
	\mathit{A}_k(b_k - b_{k+1}),\\
	  \:where\:A_k = \sum_{i=0}^k a_k$
	

	\paragraph*{Counting Rules}
	If there are \textit{m} choices for some event and \textit{n}
        choices for another event to occur and events are disjoint,
        then there are \textit{m + n} choices for either event to occur. \\
	If there are \textit{m} choices for some event and \textit{n}
        choices for another event, then there are \textit{mn} choices
        for both events.

	\paragraph*{Permutations}
	An arangement (or ordering) of distinct objects without replacement. \\
	The number of permutations of \textit{n} distinct objects is \textit{n!}. \\
	An \textit{r-permutation} of \textit{n} objects is a
        permutation of \textit{r} of the objects. \\
	The number of r-permutations of \textit{n} distinct objects is
	\[P(n, r)=\frac{n!}{(n - r)!}\]
	B is an \textit{n}-element bag with \textit{k} distinct
        elements, where each of the numbers $m_1,\ldots,m_k$ denotes
        number of occurences of each element.  The number of
        permutations of the \textit{n} elements of B is \[\frac{n!}{m_1! \ldots m_k!}\]

	\paragraph*{Combinations}
	Chosing some objects from set of objects without order.
	An \textit{r-combination} of \textit{n} distinct objects is a
        combination of \textit{r} of the objects. The number of
        \textit{r-combinations} chosen from \textit{n} distinct objects is
	\begin{center}$C(n,r)=\frac{n!}{r!(n-r)!}$ or $(x+y)^n=\sum_{k=0}^nC(n,r)x^{n-k}y^k$\end{center}
	The number of \textit{k}-element bags whose distinct elements
        are chosen from an \textit{n}-element set, where \textit{k}
        and \textit{n} are positive, is given by $C(n+k-1, k)$

	\paragraph*{Probability Distribution} A \emph{probability
          distribution} on a sample space $S$ is an assignment of
        probabilities to the points of $S$ such that the sum of all
        the probabilities is 1.
	\textbf{Probability of an Event} The \emph{probability} of an
        event $E$ is denoted by $P(E)$ and is defined by
	\begin{center}$P(E)=\sum_{x\in E}P(x)$\end{center}
	For instance, $P(S)=1$ and $P(\emptyset)=0$; $P(A\cup B)=P(A)+P(B)-P(A\cap B); P(E')=1-P(E)$
	
	\paragraph*{Conditional Probability} If $A$ and $B$ are events and
        $P(B)\neq0$, then \emph{the conditional probability of A given
          B} is denoted by $P(A\mid B)$ and defined by
	\begin{packed_list}
          \item $P(A\mid B)=\frac{P(A\cap B)}{P(B)}$
	\item $P(A\cap B)=P(B)P(A\mid B)$
	\item $P(A\cap B)=P(A)P(B\mid A)$
          \end{packed_list}

	\subsubsection*{Bayes' Theorem}
	\[P(H_i\mid E)=\frac{P(H_i\cap E)}{P(H_1\cup E)+\ldots+P(H_n\cap E)}\]
	\[P(H_i\mid E)=\frac{P(H_i\cap E)}{P(H_1)P(E\mid H_1)+\ldots+P(H_n)P(E\mid  H_n)}\]
	
	\paragraph*{5.4.3 Independent Events}
	Two events $A$ and $B$ are \emph{independent} if the following
        equation holds: $P(A \cap B)=P(A)P(B)$.
	
	\paragraph*{Binomial Distribution}
	$\binom{n}{k}p^k(1-p)^{n-k}$

	\paragraph*{Solving Recurrences by Substitution}
	Substitute occurences $f_n$ on right side of equation until pattern emerges.
	\begin{flushleft}$f_0=b_0$ and $f_n=2f_{n-1}+n$ and $f_n=2^2r_{n-2}+2(n-1)+n$\end{flushleft}
	
	\paragraph*{Solving Recurrences by Cancellation}
	Write succeeding equations for $f_n$ such that the term on the
        left side is the same as the term that contains $f$ on the
        right side of previous equation.
        \begin{flushleft}$r_n=2_{n-1}+n$ and
          $2r_{n-1}=2^2r_{n-2}+2(n-1)$
          $\ldots$ $2^{n-1}r_1=2^nr_0+2^{n-1}$
        \end{flushleft}
	Add equations, cancel like terms, replace $r_0$ by its value.
	
        \paragraph*{Comparing Rates of Growth}
        Big Oh (Big-O) can be defined as \textit{the growth rate of f is
          bounded above by the growth rate of g} if there exists positive
        numbers \textit{c} and \textit{m} such that $|f(n)| \leq
        c|g(n)| \: \textrm{for}\: n\geq m$. In this case we write $f(n)
        = O(g(n))$ and we say the $f(n)$ is \textit{big oh of $g(n)$}
        \textbf{Properties of Big Oh}
        \begin{itemize}
        \item[a.] $f(n)=O(f(n))$
        \item[b.] If $f(n)=O(g(n))$ and $ g(n)=O(h(n))$, then $f(n)=O(h(n))$
        \item[c.] If $0\leq f(n) \leq g(n)$ for all $n \geq m$, then $f(n)=O(g(n))$
        \item[d.] If $f(n)=O(g(n))$ and \textit{a} is any real number, then $af(n)=O(g(n))$
        \item[e.] If $f_1(n)=O(g(n))$ and $f_2(n)=O(g(n))$, then $f_1(n) + f_2(n) = O(g(n))$
        \item[f.] If $f_1$ and $f_2$ have nonnegative values and
          $f_1(n) = O(g_1(n)) and f_2(n) = O(g_2(n))$, then $f_1(n) +
          f_2(n) = O(g_1(n) + g_2(n))$
	\end{itemize}

\end{document}
